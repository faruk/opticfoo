\section{Anforderungsanalyse}


\subsection{VJing}

Die Pr\"asentation von visuellem Material setzt gewisse Technik vorraus, welches verschiedene Medienformate unterst\"utzt,
andere nur unter Umst\"anden und manches vielleicht gar nicht. Beispielsweise unterst\"utzt ein Videorecorder keine digitalen
Videodateien. M\"ochte man die Performance mithilfe von Videoclips realisieren und braucht Funktionen wie Zeitraffer oder
automatisches Vorw\"arts-R\"uckw\"arts-Abspielen ist eine Software gesucht die das kann. Legt man viele Videos \"ubereinander 
setzt das gewisse Rechenressourcen vorraus, mit denen das Resultat dargestellt wird.

Vor allem auf der Softwareseite sind viele Abh\"angigkeiten zu beachten. Da bei dem VRC-Konzept das Abspielen von Videoclips
nicht verloren gehen soll, die Darstellung und Animation von 3D-Inhalten zu Musik prim\"ares Ziel sind, stellt sich die Frage
mit welche Libraries in Frage kommen.

Der Bedienbarkeit kommt auch ein eigenes Feld zu. Viele VJs benutzen MIDI Controller, um die Darstellung zu steuern, da 
Potentiometer gef\"uhlvolles Einstellen von Parametern erm\"oglicht. Gleichzeitig muss ein VJ auf drastische \"Anderungen 
an der Darstellung vorbereitet sein.

\subsubsection{Hardware und Software}

Die Hardware sollte mobil sein. Ein Videoprojektor und ein Laptop sind die Grundlage f\"ur das Performen. F\"ur die 
Soundanalyse kann man direkt das Signal von einem Verst\"arker abgreifen oder \"uber ein Mikrofon aufnehmen. Gleichzeitiges 
Abspielen von mehreren Visuals auf einmal sollte m\"oglich sein. Das Anordnen der Visuals in einem dreidimensionalen
Raum erfordert Software, die das kann und auf Hardwareseite ist 3D-Beschleunigung n\"otig. 
\\
Ein Laptop, der mehrere Videoclips und 3D-Modelle auf einmal berechnet und dazu den Sound analysiert und mit der Szenerie
verkn\"upft um Animationen und Visualit\"at zu erzeugen, braucht ein gewisses Mass an Rechenkraft. Da nicht nur die 
Grafikkarte zum Rendern beansprucht wird, sondern vor allem auch die CPU f\"ur das schnelle Analysieren des Sounds und 
Ausf\"uhren von Funktionen zust\"andig ist, darf hier nicht gespaart werden. Je nach dem wie Aufw\"andig die Szenerie
gestaltet ist ergeben sich Mengen von Polygonen, die transformiert werden m\"ussen. Auch hier sollten gen\"ugend 
Leistungsreserven vorhanden sein, weshalb von Integrierten Grafikkarten abzuraten ist, da dedizierte Grafikkarten 
momentan mehr Leistung haben.

Laptops, die f\"ur Computerspiele, technisches Zeichnen, Bilder- und Videoverarbeitung  konstruiert sind 
verf\"ugen meist schon \"uber eine dedizierte Grafikkarte und eine besserer CPU als regul\"arer B\"urorechner.
\\
Hinsichtlich Kosten und Plattformwahl soll m\"oglichst freie Software verwendet werden. Eine leicht erlernbare 
Programmiersprache zur Erstellung von Visuals soll es VJs mit mit geringen Programmierkenntnissen erm\"oglichen Visuals
f\"ur das Pr\"asentationsprogramm zu programmieren. Eine freie Grafikbibliothek wie OpenGL bietet eine Grundlage f\"ur 
Rendering der Szenerie, jedoch stellt das Programmieren von OpenGL eine gro\ss e H\"urde f\"ur VJs mit nicht so tiefen
Programmierkenntnissen dar. Deher muss es die Komplexit\"at von OpenGL vereinfacht werden und Bibliotheken die auf OpenGL
aufsetzen benutzt werden.


\subsubsection{Bedienbarkeit}

F\"ur die Bedienung sollen standard Eingabeger\"ate wie Tastatur und Maus gen\"ugen. Vor allem die Einstellung der Ansicht
und Positionierung der Visuals in einem 3D-Raum muss auf einer Tastatur komfortabel m\"oglich sein - und das 
wom\"oglich stundenlang. \"Uberblendungen und Visualwechsel beinhalten viele Aktionen wie das \"Andern von Transparenzen oder
Farben und muss fl\"ussig machbar sein. Die Ansicht durch eine virtuelle Kamera erfordert eine bewegliche Kamera. Bei 
Rotationen der Kamera ist es wichtig die Orientierung nicht zu verlieren.

Eine Kombination aus Spielsteuerung f\"ur das ansteuern der Visual-Objekte und der Kamera scheint mit einer Tastatur 
realisierbar und sinnvoll. Das Anzeigen und Einstellen von Statusinformationen von Kamera,
Soundanalyse und aktiven Visuals sowie die \"Ubersicht der verf\"ugbaren Visuals erfordert eine geeignete 
Benutzerschnittstelle mit Panelen an denen Parameter abgelesen und ge\"andert werden k\"onnen.

Wichtig ist eine Reservierung von Tasten auf der Tastatur  f\"ur die Steuerung der ins Visual individuell einprogrammierten
Routinen.


\subsection{Visuals}

In dem VJ-Konzept ist der Fokus auf eine liberale Visualgestaltung ausgerichtet. Operationen wie Verschieben, Skalieren, 
Rotieren und transformation zu Musik sollen jedem Visual aufgrund seines Charakters als Objekt in einem 3D-Raum m\"oglich sein. 
Der Sound
soll von Visuals genutzt werden k\"onnen um Effekte auszul\"osen oder zu beeinflussen. Visuals bestehen bspw aus Punkten die zu 
geometrischen Formen verbunden werden k\"onnen. Vorgefertigte 3D-Modelle sollen unterst\"utzt werden. Texturen k\"onnen 
aus Bilddateien, Shaderprogrammen oder Videoclips bestehen. Durch das Programmieren der Visuals hat man 
die M\"oglichkeit eigene Bibliotheken zu benutzen und Algorithmen zur Visualisierung zu implementieren. 
 
\subsubsection{Erstellen}

F\"ur das Erstellen von Visuals ist es wichtig, dass der K\"unstler seine Visualidee verwirklichen kann. G\"angige 
Formate f\"ur Bilddateien und Videos m\"ussen unterst\"utzt werden. Dadurch hat der (VJ)K\"unstler die M\"oglichkeit
verschiedene Tools zu benutzen um Bilder oder Videos zu pr\"aparieren. Im 3D-Raum k\"onnen diese dann als Texturen auf
Fl\"achen dargestellt werden. 

Vor allem soll aber der VJ auch Punkte zu Poligonnetzen verbinden k\"onnen. So k\"onnen 3D-Modelle mit 3D-Grafikprogrammen
wie bspw Blender erstellt und in Visuals verwendet werden. Dadurch ist es auch m\"oglich vorgefertigte 3D-Modelle aus dem
Internet zu benutzen, welche von Anbietern zum Beispiel f\"ur CAD (Computer Aided Design), oder Spiele zum Download bereitgestellt 
werden. Manche Formate bieten die M\"oglichkeit Animationen zu direkt im Model zu definieren. Unabh\"angig von Formaten
m\"ussen die Inhalte auch durch selbstgeschriebene Funktionen animierbar sein. 

Mit eigens definierten Funktionen kann man auf die Skalierung, Farbe, Transparenz oder Positionierung Einfluss nehmen.
Das Soundsignal als wichtiger Bestandteil muss in Funktionen verarbeitbar sein. Durch die Analyse des Soundsignals k\"onnen 
sich Funktionen dynamisch gegenseitig aufrufen und mit der k\"unstlerischen Intention verkn\"upft werden. So kann das 
Visual programmiert werden auf B\"asse vorprogrammierte Effekte auszul\"osen.

Mit der Programmierung der Tasteneingabe kann das Visual gesteuert werden. Die Zustandsa\"nderung auf eine Tasteneingabe
muss bei der Erstellung definiert werden k\"onnen. Auch hierbei sind Funktionen die Grundlage daf\"ur.

primitive, 3d meshes, 
fertige modelle mit blender
videoclips
tastenbelegung


\subsection{Darstellung}

Ein Programm zur Darstellung der Visuals ist n\"otig um alle Visuals mit dem Soundsignal zu versorgen, Eingaben entgegen
nimmt und auf Parameter der Visuals zugreiffen kann. 


Das Programm stellt den 3D-Raum bereit in denen die Visuals geladen werden k\"onnen sowie eine \"Ubersicht \"uber
Parameter, Eingaben und Metainformationen \"uber die Zust\"ande der Visuals. Eine \"Ubersicht \"uber die Positionen
der Visuals und Kameras sowie der Ausrichtung der Kamera hilft dabei Szenen und Kameraeinstellungen zu koordinieren.

Diese Anforderungen k\"onnen in einer GUI untergebracht werden. Eine GUI fasst dabei Eingabeschnittstellen f\"ur die Visuals,
die Kameras und Soundeinstellungen zusammen, an denen die wichtige Daten \"uber den Zustand des Dargestellten abgelesen
werden k\"onnen.

Eine Ansicht durch eine Kamera wird dabei in Vollbildschirmmodus an das Ausgabeger\"at (Videoprojektor) geschickt. Eine 
andere Ansicht in den Raum erm\"oglicht das direkte Interagieren mit Visuals und Kamera. Diese Ansicht soll den
Computerspielcharakter bereitstellten wodurch direkte Manipulation an Rotation und Position m\"oglich ist.

%Dabei m\"ussen verschiedene Tasks ausgef\"uhrt werden.
%programm zur das sound verarbeitet, 
%parameter einstellung
%taskmanager, 
%Eingabe verarbeiten

%\subsubsection{3D Raum und Camera}

%Durch die Ansicht des 3D-Raums durch die virtuelle Kamera wird der 3D-Raum zum Beh\"alter f\"ur Visuals, der betrachtet wird.
%Die Positionierung der Kamera und der Visuals

%container fuer visuals
%beinhaltet cameras zur ansicht
%Camerapositionierung

\subsubsection{Performen}

Die Performance des VJs besteht aus dem Laden und Entfernen von Visuals in dem 3D-Raum. Visuals werden um die
Kamera angeordnet und bilden die Szenerie. Bei \"Anderungen in der Musik wird die Szenerie angepasst oder der
Blickwinkel auf die Szenerie ge\"andert. Musik enth\"alt oft H\"ohepunkte und viele k\"unstlerische Stilmittel.
Anhand der audialen Wahrnehmung ... balblala
laden/entfernen von visuals
ansichten aendern ohne ruckel / schnitte / camerafahrten
visuals positionieren
ueberblendungen / fading





\section{Entwurf}

\subsection{\"Uberlegungen}

\subsubsection{Soundanalyse}

\subsubsection{Bedienung}

\subsection{Spiele-Engine}

\subsubsection{Szenengraph}

\subsection{Visual-Klasse}

\subsubsection{Attribute}

\subsubsection{Methoden}



\section{Prototyp}

\subsection{Softwarewahl}

\subsection{Datenstruktur}

\subsection{Arbeitsweise}



\section{Test}

\section{Evaluierung}

\section{Verbesserungen}

\section{Resultat}
